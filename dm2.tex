\input{src/macros.tex}

\title{Optimisation \--- DM 2}
\author{Marc \textsc{Chevalier}}
\date{}

\begin{document}

\maketitle

\section*{Exercice 1}

\subsection*{a}

Soit $E$ un ensemble fini et $S=\set{S_1,\ldots,S_m}{}\subseteq \parties{E}$.

On se donne aussi $w:S\ra \RR^+$.

Pour chaque $S_i$, on introduit une variable $x_i$. On a 
\[
    \forall i \in \entiers{1}{m}, 0\leqslant x_i
\]

On veut aussi que chaque élément soit couvert :
\[
    \forall e\in E, \sum\limits_{S_i \ni e} x_i \geqslant 1
\]

Le but est de minimiser le cout total $\sum\limits_{i=1}^m x_i \cdot w(S_i)$.

\subsection*{b}

On introduit une variable $y_e$ pour chaque $e\in E$.

Le problème dual est de maximiser $\sum\limits_{e \in E} y_e$.

Avec les contraintes
\[
    \forall i\in\entiers{1}{m}, \sum\limits_{e\in S_i} y_e \leqslant w(S_i)
\]
et
\[
    \forall e\in E, y_e \geqslant 0
\]

\subsection*{c}

Une solution entière du problème primal est la liste des valeurs
\[
    \big([S_i\in c]\footnotemark\big)_{i\in\entiers{1}{m}}
\]\footnotetext{\url{https://fr.wikipedia.org/wiki/Crochet_de_Iverson}}

Pour le dual, $y_e=1$ si $e$ est choisit ou $0$ sinon. Le but est de maximiser le nombre de points choisis tel que deux points choisis ne soient pas dans le même $S_i$.

\subsection*{d}
 
On pose $M=\max\set{|s|}{s\in S}$

$f : x\mapsto Mx$ répond au problème. En effet si on prend l'optimum réel, en multipliant la solution par $M$, la condition sur $e$
\[
    \sum\limits_{s \ni e} x_s \geqslant M
\]
impose qu'il existe un des termes de la somme plus grand que 1 (par définition de $M$). On peut donc prendre la partie entière des $x_s$ et respecter le système d'origine, l'objectif de cette solution est alors inférieur à $M$ fois l'objectif du système réel optimal.

\subsection*{e}


Choisissons $S_0$ formés de $\log n$ éléments de $S$ piochés uniformément.

\[
    \forall x \in E, \forall s \in S, \proba{x \in s}{}=\frac{1}{2}
\]

\[
    \begin{aligned}
        \forall x \in E, \proba{x \in \bigcup\limits_{s\in S_0} s}{} &= \sum_{s_i in S_0}\proba{x \in s_i \text{ et } \forall t<i, x \not\in s_t }{}\\
        &= \sum_{i=1}^{\log n}{\frac{1}{2^i}}\\
        &= 1-\frac{1}{2^{\log n}}\\
        &= 1-\frac{1}{n}\\
    \end{aligned}
\]

On prend alors $\forall i, x_i=\frac{1}{n-1}\underset{+\infty}{\longrightarrow} 1$, ce qui en moyenne valide les conditions pour l'instance réelle.

Ainsi $\esp{R-SC^*}\leqslant \frac{n}{n-1}$

Pour l'instance entière :

\[
    \proba{E=\bigcup\limits_{s\in S_0} s}{}={\left(\frac{n-1}{n}\right)}^n \underset{+\infty}{\longrightarrow} \frac{1}{e}
\]

Ainsi avec probabilité non négligeable, $SC^* \leqslant (\log n) (R-SC*)$


\subsection*{f}



\section*{Exercice 2}

\subsection*{a}

Dans $D$, on a 
\[
    \begin{aligned}
        z &= v + \sum\limits_{j\in\overline{B}} c_j x_j \\
        &= v + c_s y + \sum\limits_{j\in\overline{B} \wedge s\neq j} c_j x_j\\
        &= v + c_s y
    \end{aligned}
\]

Dans $D^*$, on a 
\[
    \begin{aligned}
        z &= v + \sum\limits_{j\in B} c_j^*x_j + \sum\limits_{j\in \overline{B}} c_j^*x_j \\
        &= v + c^*_s y + \sum\limits_{i\in B} c_i^* x_i \\
        &= v + c^*_s y + \sum\limits_{i\in B} c_i^* (b_i - a_{i,s}y)
    \end{aligned}
\]

D'où
\[
    v + c_s y = v + c^*_s y + \sum\limits_{i\in B} c_i^* (b_i - a_{i,s}y)
\]

\subsection*{b}

\[
    \begin{aligned}
        v + c_s y = v + c^*_s y + \sum\limits_{i\in B} c_i^* (b_i - a_{i,s}y) &\LRa c_s y = c^*_s y + \sum\limits_{i\in B} c_i^* (b_i - a_{i,s}y)\\
        &\LRa \left(c_s - c^*_s + \sum\limits_{i\in B} c_i^*  a_{i,s}\right) y - \sum\limits_{i\in B} c_i^* b_i = 0\\
        &\LRa \left(c_s - c^*_s + \sum\limits_{i\in B} c_i^*  a_{i,s}\right) y =0\\
        &\LRa c_s - c^*_s + \sum\limits_{i\in B} c_i^*  a_{i,s} =0
    \end{aligned}
\]

\subsection*{c}

Comme $x_s$ est entrante dans $D$, on a $c_s > 0$. 

\subsection*{d}

Si $c^*_s > 0$, comme on a $s<t$, $x_s$ serait entrante car on choisit la variable de plus petit indice. Or, dans $D^*$, c'est $x_t$ qui est entrante, donc $c^*_s \leqslant 0$.

\subsection*{e}

On a 
\[
    c_s - c^*_s + \sum\limits_{i\in B} c_i^*  a_{i,s} =0
\]
soit
\[
    c^*_s - c_s = \sum\limits_{i\in B} c_i^*  a_{i,s}
\]
or $c^*_s - c_s<0$ donc $\sum\limits_{i\in B} c_i^*  a_{i,s} < 0$ donc au moins un des termes est négatif, d'où l'existence de $r\in B$ tel que $c_r^*a_{r,s} < 0$.

\subsection*{f}

Supposons que $x_r$ ne soit pas active. Cela signifie qu'elle reste basique tout au long du cycle. Par conséquent, elle a un coefficient nul dans la fonction objectif de $D^*$ : $c^*_r = 0$. Ce qui contredit$c_r^*a_{r,s} < 0$. Donc $x_r$ est nécessairement active.

\subsection*{g}
\subsection*{h}

De même qu'en \textbf{d}, comme $r\neq t$ $c_r^* \leqslant 0$ et donc, grâce à \textbf{e}, $a_{rs}>0$.

\subsection*{i}
\subsection*{j}

\section*{Exercice 3}

On introduit tout un tas de variables. Chacune est indexée par $i$ qui prend les valeurs de $\entiers{1}{4}$.

Il nous faut des variables pour
\begin{itemize}
    \item les gains : $g_i$ (non nécessairement positive)
    \item le nombre de fabricants : $f_i$ (positive)
    \item le nombre de formateurs : $t_i$ (positive)
    \item le nombre de stagiaires : $s_i$ (positive)
    \item le nombre total d'ouvrier : $o_i$ (positive)
\end{itemize}

On note $p_i$ le prix de vente d'un téléphone à la semaine $i$.

On a $p_1 = 20$, $p_2 = 18$, $p_3 = 16$ et $p_4 = 14$.

L'objectif est de maximiser $\sum\limits_{i=1}^4 g_i$.

Avec certaines contraintes :
\begin{itemize}
    \item Les stagiaires deviennent des ouvriers
        \[
            o_{i+1} =  o_i + s_i
        \]
    \item Le nombre total de fabriquant et de formateur est inférieur au nombre total d'ouvriers
        \[
            f_i+p_i \leqslant o_i
        \]
    \item Chaque formateur forme 3 stagiaires
        \[
            s_i=3\cdot t_i        
        \]
    \item Le gain est le revenu auquel on soustrait les salaires.
        \[
            g_i = 50\cdot p_i \cdot f_i-200\cdot o_i-100\cdot s_i
        \]
    \item Il faut produire au moins 20000 téléphones
        \[
            50(b_1+b_2+b_3+b_4) = 20000
        \]
    \item De façon à initialiser
        \[
            o_1 = 40
        \]
\end{itemize}

\end{document}


