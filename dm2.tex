\input{src/macros.tex}

\title{Optimisation \--- DM 2}
\author{Marc \textsc{Chevalier}}
\date{}

\begin{document}

\maketitle

\section*{Exercice 1}

\subsection*{a}

Soit $E$ un ensemble fini et $S=\set{S_1,\ldots,S_m}{}\subseteq \P(E)$.

On se donne aussi $w:S\ra \RR^+$.

Pour chaque $S_i$, on introduit une variable $x_i$. On a 
\[
    \forall i \in \llbracket 1,m\rrbracket, 0\leqslant x_i
\]

On veut aussi que chaque élément soit couvert :
\[
    \forall e\in E, \sum\limits_{S_i \ni e} x_i \geqslant 1
\]

Le but est de minimiser le cout total $\sum\limits_{i=1}^m x_i \cdot w(S_i)$.

\subsection*{b}

On introduit une variable $y_e$ pour chaque $e\in E$.

Le problème dual est de maximiser $\sum\limits_{e \in E} y_e$.

Avec les contraintes
\[
    \forall i\in\llbracket 1,m\rrbracket, \sum\limits_{e\in S_i} y_e \leqslant w(S_i)
\]
et
\[
    \forall e\in E, y_e \geqslant 0
\]

\subsection*{c}

Une solution entière du problème primal est la liste des valeurs
\[
    ([S_i\in c]\footnotemark)_{i\in\llbracket 1,m\rrbracket}
\]\footnotetext{\url{https://fr.wikipedia.org/wiki/Crochet_de_Iverson}}

\section*{Exercice 2}

\subsection*{a}

Dans $D$, on a 
\[
    \begin{aligned}
        z &= v + \sum\limits_{j\in\overline{B}} c_j x_j \\
        &= v + c_s y + \sum\limits_{j\in\overline{B} \wedge s\neq j} c_j x_j\\
        &= v + c_s y
    \end{aligned}
\]

Dans $D^*$, on a 
\[
    \begin{aligned}
        z &= v + \sum\limits_{j\in B} c_j^*x_j + \sum\limits_{j\in \overline{B}} c_j^*x_j \\
        &= v + c^*_s y + \sum\limits_{i\in B} c_i^* x_i \\
        &= v + c^*_s y + \sum\limits_{i\in B} c_i^* (b_i - a_{i,s}y)
    \end{aligned}
\]

D'où
\[
    v + c_s y = v + c^*_s y + \sum\limits_{i\in B} c_i^* (b_i - a_{i,s}y)
\]

\subsection*{b}

\[
    \begin{aligned}
        v + c_s y = v + c^*_s y + \sum\limits_{i\in B} c_i^* (b_i - a_{i,s}y) &\LRa c_s y = c^*_s y + \sum\limits_{i\in B} c_i^* (b_i - a_{i,s}y)\\
        &\LRa \left(c_s - c^*_s + \sum\limits_{i\in B} c_i^*  a_{i,s}\right) y - \sum\limits_{i\in B} c_i^* b_i = 0\\
        &\LRa \left(c_s - c^*_s + \sum\limits_{i\in B} c_i^*  a_{i,s}\right) y =0\\
        &\LRa c_s - c^*_s + \sum\limits_{i\in B} c_i^*  a_{i,s} =0
    \end{aligned}
\]

\subsection*{c}

Comme $x_s$ est entrante dans $D$, on a $c_s > 0$. 

\subsection*{d}

Si $c^*_s > 0$, comme on a $s<t$, $x_s$ serait entrante car on choisit la variable de plus petit indice. Or, dans $D^*$, c'est $x_t$ qui est entrante, donc $c^*_s \leqslant 0$.

\subsection*{e}

On a 
\[
    c_s - c^*_s + \sum\limits_{i\in B} c_i^*  a_{i,s} =0
\]
soit
\[
    c^*_s - c_s = \sum\limits_{i\in B} c_i^*  a_{i,s}
\]
or $c^*_s - c_s<0$ donc $\sum\limits_{i\in B} c_i^*  a_{i,s} < 0$ donc au moins un des termes est négatif, d'où l'existence de $r\in B$ tel que $x_r^*a_{r,s} < 0$.

\subsection*{f}



\end{document}


